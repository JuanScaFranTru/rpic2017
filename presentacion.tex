\documentclass[10pt]{beamer}

\usetheme{metropolis} \usepackage{appendixnumberbeamer}

\usepackage[utf8]{inputenc} \usepackage[spanish]{babel}

\usepackage{booktabs} \usepackage[scale=2]{ccicons}

\usepackage{pgfplots} \usepgfplotslibrary{dateplot}

\usepackage{xspace} \newcommand{\themename}{\textbf{\textsc{metropolis}}\xspace}

\title{XVII Reunión de trabajo en Procesamiento de la Información y Control}
\subtitle{Modelando el patrón temporal del vector de dengue, Chikungunya y Zika
  a partir de información satelital con redes neuronales}

\date{Septiembre de 2017}
% \titlegraphic{\hfill\includegraphics[height=1.5cm]{logo.pdf}}

\begin{document}

\maketitle

\begin{frame}{Autores}
	\begin{alertblock}{UNC - FaMAF}
    Francisco C. Trucco, Juan M. Scavuzzo
	\end{alertblock}

	\begin{alertblock}{Instituto Gulich}
    Carolina B.Tauro
	\end{alertblock}

	\begin{alertblock}{Fundación Mundo Sano}
    Alba German, Manuel Espinosa, Marcelo Abril
	\end{alertblock}
\end{frame}

\begin{frame}{Contenidos}
  \setbeamertemplate{section in toc}[sections numbered]
  \tableofcontents[hideallsubsections]
\end{frame}

\section{Introducción}

\begin{frame}{Epidemiología panorámica}

  \begin{itemize}
  \item Factores ambientales de riesgo para la salud
  %% En la epidemiología panorámica se intenta comprender cuáles son los factores
  %% del medio ambiente que significan un riesgo para la salud del ser humano.
  \item Hasta hace poco: Exclusivamente con datos de campo
  %% Hasta hace poco tiempo ese estudio ecológico se realizaba exclusivamente con
  %% información recogida en el campo, tales como temperatura, precipitaciones,
  %% humedad relativa, salinidad, vegetación, fauna, entre otros.
  \item Hoy en día: Productos satelitales
  %% Pero hoy en día, se cuenta con una herramienta adicional de gran importancia y
  %% que define un nuevo concepto en la epidemiología, el uso de información e
  %% imágenes satelitales.
  \end{itemize}

  \begin{itemize}
    \begin{center}
      \textbf{Epidemiología satelital}
    \end{center}
  \end{itemize}

\end{frame}

\begin{frame}{Dengue, Zika y Chikungunya}

  \begin{itemize}
  \item Enfermedades transmitidas por vectores
  %% Dengue, Zika y Chikungunya son enfermedades transmitidas por vectores.
  %% El dengue es una de estas enfermedades más importantes en el mundo.
  \item En América Latina el vector es el Aedes aegypti
  %% El vector de estas enfermedades en América Latina es el Aedes aegypti,
  %% mosquito peridoméstico que se cría preferentemente en contenedores
  %% artificiales.
  \item Aumento dramático de su incidencia
  %% La incidencia del dengue ha aumentado dramáticamente en las últimas décadas,
  %% con una tendencia creciente de brotes en Sudamérica en los últimos años. A
  %% esto se le suma la creciente incidencia del virus de chikungunya y zika.
  \end{itemize}

\end{frame}

\begin{frame}{Aprendizaje Automático (ML)}

  \begin{itemize}
  A computer program is said to \textbf{learn} from experience $E$ with
  respect to some class of tasks $T$ and performance measure $P$, if its
  performance at tasks in $T$, as measured by $P$, improves with experience $E$
  \end{itemize}

\end{frame}

\begin{frame}{Aprendizaje Automático (ML)}

  \begin{itemize}
  \item Eficaz para clasificaciones y regresiones de sistemas \textbf{no
    lineales}
  %% El aprendizaje automático ha resultado ser un enfoque empírico eficaz para
  %% regresiones y clasificaciones de sistemas no lineales.
  \item Ideal para problemas con:
    \begin{itemize}
    \item Importante número de observaciones
    \item Conocimiento teórico incompleto
    \end{itemize}
  %% Por esto, el aprendizaje automático es ideal para abordar aquellos problemas
  %% donde se dispone de un número importante de observaciones, pero el conocimiento
  %% teórico es aún incompleto.
  \item Tiene gran número de aplicaciones en geociencias
    \begin{itemize}
    \item Océanos
    \item Atmósfera
    \item Algoritmos de extracción de información bio-geofisica
    \end{itemize}
  \end{itemize}

\end{frame}

\begin{frame}{Enfoques del Aprendizaje Automático}

  %% Los distintos enfoques del aprendizaje automático mayormente utilizados en
  %% geociencias y sensado remoto incluyen:

  \begin{itemize}
  \item Linear Models
  \item Artificial Neural Networks
  \item Support vector machines
  \item K-Nearest Neighbour
  \item Decision trees
  \end{itemize}

\end{frame}

\begin{frame}{Scikit Learn}

  Scikit Learn es una librería de Python para aprendizaje automático que
  implementa muchos algoritmos de clasificación y regresión.

  ¿Por qué Python? ¿Por qué esta librería?

  \begin{itemize}
  \item Muy usado en el ámbito científico
  %% Python es uno de los lenguajes que han presenciado un crecimiento
  %% formidable en el ámbito académico en los últimos años además de ser uno de
  %% los más usados
  \item Tanto Python como Scikit Learn son software libre
  \item Hay una comunidad muy activa
  \item Fáciles de usar
  \end{itemize}

  %% Usamos el envi para obtener resultados de manera exploratoria
  %% Python es mucho más flexible para este tipo de problemas
  %% Uno de los objetivos de este trabajo fue usar algoritmos y herramientas
  %% \textbf{off-the-shelf}

\end{frame}

%% \begin{frame}{iRace}
%%   - Algunos modelos dependen de parámetros. Por ejemplo: [decision tree]
%%   Encontrar la mejor configuración de esos parámetros para resolver un problema
%%   constituye todo un desafío. Explicar maaaaomeno por qué.
%%   - iRace (Iterated Racing for Automatic Algorithm Configuration).
%%   Fue utilizada para encontrar las mejores configuraciones de parámetros dadas
%%   las instancias del problema. Queremos aclarar que está implementada en R y
%%   también es software libre.
%% \end{frame}

\section{Área de estudio y Datos de Campo}

\begin{frame}{Área de estudio y Datos de Campo}

  %% Ponemos fotos de tartagal, salta TODO

  %% La Ciudad de Tartagal se encuentra en la base de las sierras subandinas
  %% argentinas en la provincia de Salta. La ciudad está rodeada de bosques nativos
  %% subtropicales y cultivos. El clima es subtropical.

\end{frame}

\section{Modelos}

\begin{frame}{Variables}

  %% Después mostrar el heatmap de todas las variables sin lag. TODO

  %% El uso de ovitrampas constituye un método efectivo para proporcionar datos
  %% útiles sobre la distribución espacial y temporal del Aedes aegypti.
  %% Se usaron productos satelitales que se analizaron influyentes para el
  %% medioambiente del mosquito según análisis de nuestro equipo de trabajo.

\end{frame}

\begin{frame}{Variables}

  %% De todas las variables ambientales recolectadas, solo algunas fueron
  %% seleccionadas como entradas del modelo.

  % Mostrar el heatmap de todas las variables seleccionadas. TODO

  %% Cabe aclarar que futuros trabajos pueden mejorar la selección de
  %% características para obtener mejores resultados.

\end{frame}

\begin{frame}{Modelos}
  %% TODO Imágenes de modelos lineales
\end{frame}

\begin{frame}{Modelos}
  \begin{center}
    \textbf{¿Son las regresiones lineales las más adecuadas para modelar este
      tipo de problemas?}
  \end{center}
\end{frame}

\section{Resultados}

\begin{frame}{Resultados}
  %% TODO Poner las imágenes
\end{frame}

\section{Conclusión}

\begin{frame}{Conclusión}
  \begin{itemize}
    \item No Lineales vs Lineales

    \item Herramientas off-the-shelf
      %% Los modelos no lineales funcionan mejor que los modelos lineales.
      %% Aunque los modelos no lineales involucran una mayor complejidad a la
      %% hora de implementarlos, existen muchas herramientas off-the-shelf que
      %% facilitan esta tarea.

    \item Técnicas pocas veces utilizadas en problemáticas epidemiológicas
      %% La gran perspectiva en la utilización de este tipo de herramientas en
      %% este campo. Introducir técnicas sofisticadas como las de aprendizaje
      %% automático, pocas veces utilizadas en problemáticas epidemiológicas.

    \item Modelos operativos en el ámbito de la epidemiología
      %% La intención es que con estas herramientas se pueda desarrollar un
      %% sistema que pueda ser utilizado para llevar a cabo políticas de salud
      %% pública. El trabajo fue proyectado de manera que pudiera ser articulado
      %% con otros esfuerzos tendientes a avanzar en las direcciones
      %% establecidas por un proyecto institucional cuyo objetivo es abordar la
      %% problemática de la epidemiología.
  \end{itemize}
\end{frame}

\begin{frame}{Trabajos futuros: Cómo mejorar los resultados}
      \begin{itemize}
        \item Aumentando la cantidad de datos
        \item Haciendo una mejor selección de características
        \item Entrenando sobre otras ciudades
        %% Los dos primeros mejoran la capacidad predictiva del modelo
        %% El último permite que los modelos puedan ser aplicados a otras
        %% regiones geográficas.
      \end{itemize}
\end{frame}

\begin{frame}[standout]
  ¿Preguntas?
\end{frame}

\appendix

\begin{frame}{Referencias}

[1] S. Arthur. 1958. ‘Some studies in machine learning using the game of
checkers', IBM Journal of Research and Development, Volume:44 Issue:1.2

Available from: IEEE Digital Explore Library

[2] T. Mitchell. 1997. ‘Machine Learning’ [pdf]

Available at \url{https://www.cs.swarthmore.edu/~meeden/cs63/f11/ml-intro.pdf}

  \url{https://www.researchgate.net/profile/Alfonso_Rodriguez-Morales/publication/265167661_Ecoepidemiologia_y_epidemiologia_satelital_nuevas_herramientas_en_salud_publica_y_su_utilidad_en_la_leishmaniasis_tegumentaria_americana/links/5403593a0cf2bba34c1c225d/Ecoepidemiologia-y-epidemiologia-satelital-nuevas-herramientas-en-salud-publica-y-su-utilidad-en-la-leishmaniasis-tegumentaria-americana.pdf}

\end{frame}

\end{document}

%% \begin{frame}{Blocks}
%%   Three different block environments are pre-defined and may be styled with an
%%   optional background color.

%%   \begin{block}{Default}
%%     Block content.
%%   \end{block}

%%   \begin{alertblock}{Alert}
%%     Block content.
%%   \end{alertblock}

%%   \begin{exampleblock}{Example}
%%     Block content.
%%   \end{exampleblock}
%%   \stepcounter{beamerpauses}
%%   \begin{itemize}[<+->]
%%     \item Hola
%%     \item Chau
%%   \end{itemize}
%% \end{frame}
