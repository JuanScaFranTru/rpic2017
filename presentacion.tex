\documentclass[10pt]{beamer}

\usetheme{metropolis}
\usepackage{appendixnumberbeamer}

\usepackage[utf8]{inputenc}
\usepackage[spanish]{babel}

\usepackage{booktabs}
\usepackage[scale=2]{ccicons}

\usepackage{pgfplots}
\usepgfplotslibrary{dateplot}

\usepackage{xspace}
\newcommand{\themename}{\textbf{\textsc{metropolis}}\xspace}

\title{XVII Reunión de trabajo en Procesamiento de la Información y Control}
\subtitle{Modelando el patrón temporal del vector de dengue, Chikungunya y Zika a
  partir de información satelital con redes neuronales}

\date{Septiembre de 2017}
% \titlegraphic{\hfill\includegraphics[height=1.5cm]{logo.pdf}}

\begin{document}

\maketitle

\begin{frame}{Autores}
	\begin{alertblock}{UNC - FaMAF}
    Francisco C. Trucco, Juan M. Scavuzzo
	\end{alertblock}

	\begin{alertblock}{Instituto Gulich}
    Carolina B.Tauro
	\end{alertblock}

	\begin{alertblock}{Fundación Mundo Sano}
    Alba German, Manuel Espinosa, Marcelo Abril
	\end{alertblock}
\end{frame}

\begin{frame}{Contenidos}
  \setbeamertemplate{section in toc}[sections numbered]
  \tableofcontents[hideallsubsections]
\end{frame}

\section{Introducción}

\begin{frame}{Ecoepidemiología y epidemiología satelital}

Los cambios e interacciones del medio ambiente tienen una fuerte influencia
sobre los factores que determinan diversas enfermedades. Esto nos da la
oportunidad de estudiar dichos factores ambientales para entender cómo
condicionan las enfermedades humanas.

De allí surge la ecoepidemiología, en la cual se intenta comprender cuáles son
los factores del medio ambiente que significan un riesgo para la salud del ser
humano.

Hasta hace poco tiempo ese estudio ecológico se realizaba exclusivamente con
información recogida en el campo, tales como temperatura, precipitaciones,
humedad relativa, salinidad, vegetación, fauna, entre otros.

Pero hoy en día, se cuenta con una herramienta adicional de gran importancia y
que define un nuevo concepto en la epidemiología, el uso de información e
imágenes satelitales.

Estas herramientas abren una subespecialidad de la ecoepidemiología: la llamada
epidemiología satelital.

La epidemiología satelital puede ser pensada como una aplicación de segunda
generación de la información sensada remotamente donde el objetivo (i.e., el
vector o el huésped) no se detecta directamente de la imagen satelital. 

\end{frame}

\begin{frame}{Dengue}

  El dengue es una de las enfermedades transmitidas por vectores más importantes
  en el mundo.

  El vector de esta enfermedad en América Latina es el Aedes aegypti, mosquito
  peridoméstico que se cría preferentemente en contenedores artificiales.

  La incidencia del dengue ha aumentado dramáticamente en las últimas décadas,
  con una tendencia creciente de brotes en Sudamérica en los últimos años. A
  esto se le suma la creciente incidencia del virus de chikungunya y zika, cuyo
  principal vector es la misma especie de mosquito.

  El uso de ovitrampas constituye un método efectivo para proporcionar datos
  útiles sobre la distribución espacial y temporal del Aedes aegypti y otras
  especies de mosquitos que habitan contenedores, lo que permite obtener un
  mejor conocimiento de su actividad.

\end{frame}

\begin{frame}{Aprendizaje Automático (ML)}

El aprendizaje automático ha resultado ser un enfoque empírico eficaz para
regresiones y clasificaciones de sistemas no lineales. Por esto, el aprendizaje
automático es ideal para abordar aquellos problemas donde se dispone de un
número importante de observaciones, pero el conocimiento teórico es aún
incompleto.

Las técnicas del aprendizaje automático han demostrado ser útiles para un gran
número de aplicaciones en geociencias, tanto para tierra, océanos y atmósfera, y
en algoritmos de extracción de información bio-geofisica.

\end{frame}

\begin{frame}{Enfoques del Aprendizaje Automático}

Los distintos enfoques del aprendizaje automático mayormente utilizados en
geociencias y sensado remoto incluyen [7]:

\begin{itemize}
\item artificial neural networks (ANN)
\item support vector machines (SVM)
\item self-organizing map (SOM)
\item decision trees (DT)
\item random forests (RF)
\item genetic algorithms (GA)
\end{itemize}

Esta área del Geoscience and Remote Sensing (GRS) es relativamente nueva y
extremadamente prometedora ([7], [8]).

En particular, las Redes Neuronales Artificiles, son ampliamente utilizadas para
clasificación, pero también para predicciones de series de tiempo ([9], [10],
[11], [12]).

De hecho, una exploración en la base bibliográfica scopus nos arroja una
cantidad de más de 4000 publicaciones que incluyen remote sensing y neural
network con 311 publicaciones para el 2016.

\end{frame}

\begin{frame}{Scikit Learn}
  Scikit Learn es una librería de Python para aprendizaje automático que
  implementa muchos algoritmos de clasificación y regresión.  

  ¿Por qué esta librería? ¿Por qué Python?

  \begin{itemize} 
  \item Python es uno de los lenguajes que han presenciado un crecimiento
    formidable en el ámbito académico en los últimos años además de ser uno de
    los más usados.
  \item Tanto Python como Scikit Learn son software libre.
  \item Hay una comunidad muy activa.
  \item Facilita el trabajo interdisciplinario debido a la baja complejidad de
    la herramienta.
  \end{itemize}

  Uno de los objetivos de este trabajo fue usar algoritmos y herramientas
  \textbf{off-the-shelf}.

\end{frame}

\begin{frame}{iRace}

  Como veremos luego, algunos modelos dependen de parámetros. Encontrar la mejor
  configuración de esos parámetros para resolver un problema constituye todo un
  desafío.

  Una herramienta llamada iRace (Iterated Racing for Automatic Algorithm
  Configuration) fue utilizada para realizar encontrar las mejores
  configuraciones de parámetros dadas las instancias del problema.

  Queremos aclarar que está implementada en R y también es software libre.

\end{frame}

\begin{frame}{Área de estudio y Datos de Campo}

La Ciudad de Tartagal se encuentra en la base de las sierras subandinas
argentinas en la provincia de Salta. La ciudad está rodeada de bosques nativos
subtropicales y cultivos. El clima es subtropical, con una temperatura media
anual de unos 23; con máximas promedio de 39 y mínimas promedio de 9. La
precipitación anual es de unos 1100 mm, con una estación seca de junio a octubre
y una estación húmeda de noviembre a mayo con una precipitación mensual media de
160 mm mensuales. El área urbana de la ciudad de Tartagal cubre aproximadamente
15 $km^2$ y está compuesta por unas 18.000 viviendas.

%% Acá deberíamos poner una imagen y tratar de decir mucho menos que lo que está
%% escrito.

\end{frame}

\begin{frame}{Variables}
Para caracterizar el hábitat de los mosquitos se obtuvieron variables derivadas
de imágenes de satelitales (principalmente del sensor MODIS), a saber: NDVI,
NDWI, LST y la precipitación.

El índice de vegetación de diferencia normalizada (NDVI) y el índice diferencial
de agua normalizada (NDWI) se obtuvo a partir del producto del sensor MODIS
MOD13Q1 (16 días) con una resolución espacial de 250m.

Para el LST se seleccionó el producto satelital MOD11A2, que se almacena en una
grilla de 1 km como los valores medios de LSTs de cielo claro durante un período
de 8 días. El producto MOD11A2 está compuesto por LSTs diurnos y nocturnos que
se pueden considerar representativos de las temperaturas máxima y mínima

Además, la precipitación local se obtuvo de The Tropical Rainfall Measuring
Mission (TRMM) [33], misión conjunta de la NASA y la Agencia de Japón que fue
lanzada en 1997 para estudiar las precipitaciones.

% Acá podriamos hacer un cuadro diciendo de donde sacamos cada variable para no
% tener que hablar tanto.
% Después mostrar el heatmap de todas las variables sin lag.

Para extraer los indicadores geofísicos de las imágenes descriptas se definieron
dos áreas de 85 ha (Fig. 2) y se calcularon los valores medios para todas las
variables. La primera área está ubicada dentro de la ciudad (Área Urbana) y la
segunda abarca la vegetación nativa que rodea la ciudad (Área Rural).

La hipótesis ya utilizada en [34] es que las condiciones ambientales (NDWI,
NDVI, LST) que rodean la ciudad podrían ser buenas indicadoras de los efectos
climáticos y ambientales que están afectando los índices larvales dentro de la
ciudad.

\end{frame}

\begin{frame}{Selección de Variables}

De todas las variables ambientales recolectadas, solo algunas fueron
seleccionadas como entradas del modelo.

Esta selección se basó tanto en un análisis de las propiedades biológicas
intrínsecas del problema como en un análisis de la correlación de las variables
ambientales obtenidas con imágenes satelitales. El procedimiento se describe con
más detalle en [27].

Las variables seleccionadas resultaron ser: NDVI, LST nocturna y NDWI de la
región rural y urbana con un lag de 1 semana, LST nocturna urbana con un lag de
2 semanas, LST diurna rural y urbana con un lag de 3 semanas, TRMM con un lag de
3 semanas, además de número de días fríos y la temperatura correspondiente a la
región rural.

% Mostrar el heatmap de todas las variables seleccionadas sin lag (para evitar
% hablar tanto).

Cabe aclarar que futuros trabajos pueden mejorar la selección de características
para obtener mejores resultados.

\end{frame}

\begin{frame}{Modelos}

\end{frame}

\begin{frame}{Resultados}

\end{frame}

\begin{frame}{Conclusiones}

\end{frame}

\begin{frame}[standout]
  ¿Preguntas?
\end{frame}

\appendix

\begin{frame}{Referencias}

[1] S. Arthur. 1958. ‘Some studies in machine learning using the game of
checkers', IBM Journal of Research and Development, Volume:44 Issue:1.2

Available from: IEEE Digital Explore Library

[2] T. Mitchell .1997. ‘Machine Learning’ [pdf]

Available at \url{https://www.cs.swarthmore.edu/~meeden/cs63/f11/ml-intro.pdf}

  \url{https://www.researchgate.net/profile/Alfonso_Rodriguez-Morales/publication/265167661_Ecoepidemiologia_y_epidemiologia_satelital_nuevas_herramientas_en_salud_publica_y_su_utilidad_en_la_leishmaniasis_tegumentaria_americana/links/5403593a0cf2bba34c1c225d/Ecoepidemiologia-y-epidemiologia-satelital-nuevas-herramientas-en-salud-publica-y-su-utilidad-en-la-leishmaniasis-tegumentaria-americana.pdf}

\end{frame}

\end{document}

%% \begin{frame}{Blocks}
%%   Three different block environments are pre-defined and may be styled with an
%%   optional background color.

%%   \begin{block}{Default}
%%     Block content.
%%   \end{block}

%%   \begin{alertblock}{Alert}
%%     Block content.
%%   \end{alertblock}

%%   \begin{exampleblock}{Example}
%%     Block content.
%%   \end{exampleblock}
%%   \stepcounter{beamerpauses}
%%   \begin{itemize}[<+->]
%%     \item Hola
%%     \item Chau
%%   \end{itemize}
%% \end{frame}
